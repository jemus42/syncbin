\documentclass[a4paper, 12pt, bibtotoc]{scrartcl} % bibtotoc für Literaturverzeichnis in ToC
\usepackage[utf8]{inputenc} 
\usepackage[T1]{fontenc}
\usepackage[ngerman]{babel}
\usepackage[square]{natbib}
\usepackage{pdfpages}
\usepackage{scrpage2}
\usepackage[hyphens]{url} % Für URLs (duh.)
\usepackage{setspace}
\usepackage{url}
\usepackage{microtype} % Macht Typographievodoo ¯\_(ツ)_/¯
%\setlength{\headheight}{1.1\baselineskip}
\usepackage{hyperref} % Immer erst am Ende der Pakete
%\usepackage[babel,german=quotes]{csquotes} 
\clubpenalty = 10000
\widowpenalty = 10000
%\setlength{\headheight}{15pt}

%% META %%
\titlehead{University - Fachbereich ?? - Blah, B.A.}
%\subject{}
\title{Title}
%\subtitle{}
\date{21.07.13 \\ }
\author{Name \\ BA  \\ Matrikelnr. xxxxxxx \\ mail &&}
\publishers{&& Seminar:  (WiSe XX/XY) \\ Date \\ VAK  \\ Seminarleitung:}

%% Doc options
\onehalfspacing

%% Content %%
\begin{document}
\pagestyle{scrheadings} 
\maketitle 
#\includepdf{Deckblatt} %% Word-Deckblatt statt LaTeX-Deckblatt, the shame.
\clearpage
\tableofcontents
\thispagestyle{empty}  % Anscheinend nötig um Seitenzahlen erst später beginnen zu lassen
\pagebreak 
\setcounter{page}{1}



%% Bibliography %%
\clearpage
\bibliography{sources} 
\bibliographystyle{apalike-url.bst}
\end{document}
